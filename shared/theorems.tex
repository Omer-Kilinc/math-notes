% ============================================
% THEOREM-LIKE ENVIRONMENTS
% ============================================
% All numbered together within chapters
% Format: Theorem 1.1, Lemma 1.2, Definition 1.3, etc.

\theoremstyle{plain}  % Italic text, used for theorems
\newtheorem{theorem}{Theorem}[chapter]
\newtheorem{lemma}[theorem]{Lemma}
\newtheorem{proposition}[theorem]{Proposition}
\newtheorem{corollary}[theorem]{Corollary}
\newtheorem{conjecture}[theorem]{Conjecture}
\newtheorem{claim}[theorem]{Claim}

% ============================================
% DEFINITION-STYLE ENVIRONMENTS
% ============================================
\theoremstyle{definition}  % Upright text
\newtheorem{definition}[theorem]{Definition}
\newtheorem{example}[theorem]{Example}
\newtheorem{exercise}[theorem]{Exercise}
\newtheorem{problem}[theorem]{Problem}
\newtheorem{question}[theorem]{Question}
\newtheorem{algorithm}[theorem]{Algorithm}

% ============================================
% REMARK-STYLE ENVIRONMENTS
% ============================================
\theoremstyle{remark}  % Upright text, less emphasis
\newtheorem{remark}[theorem]{Remark}
\newtheorem{note}[theorem]{Note}
\newtheorem{observation}[theorem]{Observation}
\newtheorem{notation}[theorem]{Notation}
\newtheorem{convention}[theorem]{Convention}
\newtheorem{recall}[theorem]{Recall}

% ============================================
% UNNUMBERED VERSIONS
% ============================================
% For special cases where you don't want numbering
\newtheorem*{theorem*}{Theorem}
\newtheorem*{lemma*}{Lemma}
\newtheorem*{proposition*}{Proposition}
\newtheorem*{corollary*}{Corollary}
\newtheorem*{definition*}{Definition}
\newtheorem*{example*}{Example}
\newtheorem*{remark*}{Remark}
\newtheorem*{note*}{Note}

% ============================================
% SOLUTION ENVIRONMENT
% ============================================
\newenvironment{solution}
  {\begin{proof}[Solution]}
  {\end{proof}}